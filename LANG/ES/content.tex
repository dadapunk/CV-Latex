\cvsection{Conocimientos}
\begin{minipage}{0.33\linewidth}		
    \pointskill{}{Java}{4}
    \pointskill{}{Springboot}{3}
    \pointskill{}{J2EE}{2}
    \pointskill{}{JUnit}{3}
    \pointskill{}{Python}{3}
\end{minipage}%
\hfill%
\begin{minipage}{0.33\linewidth}
    \pointskill{}{Django}{1}
    \pointskill{}{Git}{4}
    \pointskill{}{Docker}{3}
    \pointskill{}{Microservices}{1}
    \pointskill{}{GCP}{3}
\end{minipage}%
\hfill%
\begin{minipage}{0.33\linewidth}
    \pointskill{}{Linux}{5}
    \pointskill{}{Bash}{2}
    \pointskill{}{MySQL}{4}
    \pointskill{}{PostgreSQL}{3}
    \pointskill{}{NoSQL}{1}
\end{minipage}
\cvsection{Experiencia}
\begin{cvtable}
    \cvitem{11, 2020 -- 01, 2021}{\color{cvsectioncolor}Consultor}{\textbf{ILO Consultor, CL}}{Desarrollé una solución de web scraping automatizada en Python B4s para capturar datos de portales públicos de empleo, lo que permitió a la empresa obtener información valiosa sobre el mercado laboral.}
    \cvitem{07, 2019 -- 01, 2020}{\color{cvsectioncolor}Consultor SAPUI5 -IoT}{\textbf{SCLOUD, CL}}{Diseñé y desarrollé prototipos de IoT y dashboards personalizados para un cliente del sector vinícola utilizando tecnologías como ESP32 y SAPUI5, lo que mejoró la eficiencia de su proceso de producción y le permitió tomar decisiones informadas en tiempo real.}
    \cvitem{05, 2016 -- 11, 2017}{\color{cvsectioncolor}Ingeniero de desarrollo}{\textbf{Karibu, CL}}{Desarrollé un sistema de punto de venta en desktop utilizando Java Swing y SQL, lo que permitió mejorar la eficiencia de las operaciones y optimizar el seguimiento de las ventas. También implementé el uso de reportes con Jasper, lo que facilitó el monitoreo y análisis de los datos de venta.}
    \cvitem{06, 2015 -- 02, 2016}{\color{cvsectioncolor}QA, Pruebas}{\textbf{Analyze, Cl}}{Realicé pruebas de software con PyTest y creé manual de usuario para mejorar la experiencia del usuario final en una empresa agrícola.
    Identifiqué y documenté más de 30 errores críticos en el software, lo que permitió a la empresa corregirlos antes del lanzamiento del producto, ahorrando tiempo y dinero en el proceso de corrección.}
\end{cvtable}
\cvsection{Formación}
\subsection{IT}
\begin{cvtable}
    \cvitemmedium{02, 2023}{\href{https://www.ironhack.com}{\color{cvsectioncolor}\small BackEnd Development Bootcamp (Java SpringBoot)}}{\textbf{IronHack}}
    \cvitemmedium{11, 2022}{\href{https://www.credly.com/badges/a43cfd6e-2c81-4bfe-9762-78a0256abf19/public_url}{\color{cvsectioncolor}\small Soporte de TI de Google}}{\textbf{Coursera}}
    \cvitemmedium{11, 2022}{\href{https://s3-us-west-2.amazonaws.com/udacity-printer/production/certificates/6d8cbfdf-2823-4946-a27a-2ebafce69563.pdf}{\color{cvsectioncolor}\small AWS Machine Learning Foundations}}{\textbf{Udacity}}
    \cvitemmedium{12, 2020}{\href{https://www.duoc.cl/wp-content/uploads/2021/10/INGE-INFORMATICA.pdf}{\color{cvsectioncolor}\small Ingeniería Informática}}{\textbf{DuocUC, Chile}}
    \cvitemmedium{06, 2015}{\href{https://www.duoc.cl/wp-content/uploads/2021/10/ANALISTA-PROG.pdf}{\color{cvsectioncolor}\small Analista Programador}}{\textbf{DuocUC, Chile}}
\end{cvtable}	
\subsection{Otras}
\begin{cvtable}
    \cvitemmedium{05, 2011}{\href{https://www.uchile.cl/carreras/57017/composicion-musical}{\color{cvsectioncolor}\small Composición}}{\textbf{Universidad de Chile, Chile}}
    \cvitemmedium{12, 2008}{\href{https://www.uchile.cl/carreras/4954/teoria-de-la-musica}{\color{cvsectioncolor}\small Teoría de la Música}}{\textbf{Universidad de Chile, Chile}}
\end{cvtable}
\profilesection{Perfil}
\aboutme{Profesional de TI responsable, analítico y metódico con experiencia en proyectos
que incluyen diferentes lenguajes, tecnologías y arquitecturas (web, móvil, IoT). 
Experiencia en la construcción y programación de hardware para interfaces de sonido.
Me apasiona la creación y exploración de herramientas tecnológicas que agilicen y faciliten 
procesos, mejoren la calidad de vida y contribuyan a comprender mejor nuestra realidad.}
\profilesection{Objetivos}
\aboutme{Busco seguir desarrollando mi carrera en el ámbito de la tecnología, 
explorando nuevas herramientas y tecnologías emergentes y aplicándolas en proyectos innovadores que tengan un 
impacto positivo en la sociedad. Aspiro a liderar equipos de desarrollo y colaborar con la comunidad 
de código abierto en proyectos relevantes.}